\documentclass[../doc.tex]{subfiles}
\begin{document}
\subsection{Tautologías usadas en la demostración}
\begin{enumerate}
  \item Transformación:  $p \rightarrow q \equiv \neg p \lor q$
  \item Distributividad: $p \lor (q \land r) \equiv (p \lor q) \land (p \lor r)$
  \item Distributividad: $p \land (q \lor r) \equiv (p \land q) \lor (p \land r)$
  \item Doble negación: $\neg\neg p \equiv p$
\end{enumerate}

\subsection{Demostración}
Usando inducción demostraremos que si $L(P)$ es un lenguaje proposicional, la fórmula $(\bigvee _{i=1}^{n} X_i) \rightarrow Y $ es lógicamente equivalente a $\bigwedge _{i=1}^{n} (X_i \rightarrow Y)$,
para todo $n \ge 1$.
\subsubsection*{\emph{B.I}}
\emph{n=1}
\[X_1 \rightarrow Y \equiv X_1 \rightarrow Y\]
\emph{n=2}
\[(X_1 \lor X_2) \rightarrow Y \qquad (X_1 \rightarrow Y) \land (X_2 \rightarrow Y) \]
\[\neg(X_1 \lor X_2) \lor Y \qquad (\neg X_1 \lor Y) \land (\neg X_2 \lor Y) \]
\[(\neg X_1 \land \neg X_2) \lor Y \quad \equiv \quad (\neg X_1 \land \neg X_2) \lor Y \]

\subsubsection*{\emph{H.I}}
\[ (\bigvee _{i=1}^{n} X_i) \rightarrow Y \equiv \bigwedge _{i=1}^{n} (X_i \rightarrow Y) \quad \forall n \ge 1\]

\subsubsection*{\emph{T.I}}
Para demostrar que lo propuesto en el enunciado se cumple, podemos usar el $PIS$ en lo siguiente:


\emph{Empezaremos por la fórmula lógica de la derecha}
\[ (\bigvee _{i=1}^{n+1} X_i) \rightarrow Y \equiv \bigwedge _{i=1}^{n+1} (X_i \rightarrow Y) \]

\emph{Aplicamos propiedad de la "ANDatoria"}
\[ \bigwedge _{i=1}^{n+1} (X_i \rightarrow Y) = \bigwedge _{i=1}^{n} (X_i \rightarrow Y) \land (X_{n+1} \rightarrow Y) \]

\emph{Por H.I}
\[ \bigwedge _{i=1}^{n} (X_i \rightarrow Y) \land (X_{n+1} \rightarrow Y) = ((\bigvee _{i=1}^{n} X_i) \rightarrow Y) \land (X_{n+1} \rightarrow Y) \] 

\emph{Aplicamos Transformación}
\[ ((\bigvee _{i=1}^{n} X_i) \rightarrow Y) \land (X_{n+1} \rightarrow Y) = (\neg(\bigvee _{i=1}^{n} X_i) \lor Y) \land (\neg X_{n+1} \lor Y) \]

\emph{Aplicamos Distributividad}
\[ (\neg(\bigvee _{i=1}^{n} X_i) \lor Y) \land (\neg X_{n+1} \lor Y) = (\neg(\bigvee _{i=1}^{n} X_i) \land \neg X_{n+1} ) \lor Y\]

\emph{Aplicamos Doble negación}
\[ (\neg(\bigvee _{i=1}^{n} X_i) \land \neg X_{n+1} ) \lor Y = \neg\neg(\neg(\bigvee _{i=1}^{n} X_i) \land \neg X_{n+1} ) \lor Y\] 
\[ \neg\neg(\neg(\bigvee _{i=1}^{n} X_i) \land \neg X_{n+1} ) \lor Y = \neg((\bigvee _{i=1}^{n} X_i) \lor  X_{n+1} )\lor Y \]

\emph{Aplicamos propiedad de la "ORatoria"}
\[ \neg((\bigvee _{i=1}^{n} X_i) \lor  X_{n+1}) \lor Y = \neg(\bigvee _{i=1}^{n+1} X_i) \lor Y \]

\emph{Aplicamos Transformación}
\[ \neg(\bigvee _{i=1}^{n+1} X_i) \lor Y = (\bigvee _{i=1}^{n+1} X_i) \rightarrow Y\]

\emph{Ahora es claro que:}
\[ (\bigvee _{i=1}^{n+1} X_i) \rightarrow Y \equiv \bigwedge _{i=1}^{n+1} (X_i \rightarrow Y) \]

\end{document}
