\documentclass[../doc.tex]{subfiles}
\begin{document}

Demostraremos lo siguiente:
\[ n_1 + n_2 + ... + n_k = n \]
\[ n_1^2 + n_2^2 + ... + n_k^2 \leq (n-k+1)^2 + k -1\]
La demostración se hará con una inducción sobre k, es decir, el número de 
elementos dentro de la suma.

\subsubsection*{\emph{B.I.}}
$k=0$
\[0 \leq (0 - 0 +1)^2 + 0 -1 = 0\]
$k=1$
\[ n_1^2 \leq (n_1^2 - 1 + 1)^2 + 1 - 1 = n_1^2\]
$k=2$
\[ n_1^2 + n_2^2 \leq (n_1 + n_2 - 2 + 1)^2 + 2 -1  \]
\[ n_1^2 + n_2^2 \leq n_1^2 + 2n_1n_2+n_2^2-2n_1-2n_2+2 \]
\hspace*{0pt}\hfill $/ - n_1^2 - n_2^2$
\[ 0 \leq  2n_1n_2-2n_1-2n_2+2 \]
\[ 0 \leq 2(n_1 - 1)(n_2 - 1) \]
\[ 0 \leq (n_1 - 1)(n_2 -1) \]

\[ 1 \leq n_1 = 0 \leq n_1 - 1  \]
\[ 1 \leq n_2 = 0 \leq n_2 -1  \]
\[ \therefore  0 \leq (n_1 - 1)(n_2 -1)\]


\subsubsection*{\emph{H.I.}}
Ahora asumiremos lo siguiente como correcto.
\[ n_1^2 + n_2^2 + ... + n_k^2 \leq (n-k+1)^2 + k -1\]


\subsubsection*{\emph{T.I.}}
PDQ:
\[ n_1^2 + n_2^2 + ... + n_k^2 + n_{k+1}^2 \leq ((n + n_{k+1}) - (k+1) + 1)^2 + (k+1)-1 \]
\[ n_1^2 + n_2^2 + ... + n_k^2 + n_{k+1}^2 \leq ((n + n_{k+1}) - k)^2 + k \]
-\[   n_1^2 + n_2^2 + ... + n_k^2 \leq (n-k+1)^2 + k -1\]
=\[ n_{k+1}^2 \leq (n+n_{k+1} - k)^2 + k - ((n-k+1)^2 + k -1) \]
\[ n_{k+1}^2 \leq (n+n_{k+1})^2 - 2k(n+n_{k+1}) + k^2 + k - ((n-k)^2 + 2(n-k) + 1+ k -1) \]
\[ n_{k+1}^2 \leq (n+n_{k+1})^2 - 2k(n+n_{k+1}) + k^2 - ((n-k)^2 + 2(n-k)) \]
\[ n_{k+1}^2 \leq n^2 + 2nn_{k+1} + n_{k+1}^2 - 2kn -2kn_{k+1} + k^2  - (n^2 - 2kn +k^2 + 2n - 2k)) \]
\[ n_{k+1}^2 \leq 2nn_{k+1} + n_{k+1}^2 -2kn_{k+1} - ( 2n - 2k)) \]
\hspace*{0pt}\hfill $/ -n_{k+1}^2 $
\[ 0 \leq 2nn_{k+1} -2kn_{k+1} - 2n + 2k \]
\hspace*{0pt}\hfill $/ \times\frac{1}{2} $
\[ 0 \leq nn_{k+1} -kn_{k+1} - n + k \]
\[ 0 \leq n_{k+1}(n-k) - n + k \]
\[ 0 \leq n_{k+1}(n-k) - (n - k) \]
\[ 0 \leq n_{k+1}(n-k) - (n - k) \]
\[ 0 \leq (n-k)( n_{k+1}-1) \]

Bajo las restricciones del problema, $ n_{k+1} $ siempre es mayor que 1, por lo 
tanto $ (n_{k+1}-1) $ siempre es mayor o igual a 0. 

\[ n_{k+1} \geq 1 \]
\hspace*{0pt}\hfill $/ -1 $
\[ n_{k+1} - 1 \geq 0 \]

Ademas, n es la suma de todos los $ n_{i} $, que a su vez, son mayores o
iguales a 1, menos k, que es la cantidad de elementos que se sumaran, lo que
nunca sera mayor a la suma, bajo la restriccion de que  $ n_{i} >= 1 $, por lo
tanto $ (n-k) $ tambien es positivo. Finalmente $ (n-k)( n_{k+1}-1) $ es la
multiplicacion de dos numeros positivos o 0, por lo tanto siempre sera mayor o
igual a 0. Queda demostrada la \emph{T.I.}


\end{document}
