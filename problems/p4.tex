\documentclass[../doc.tex]{subfiles}
\begin{document}
\subsection{Propiedad}
\[\sum_{p=1}^{k} pH_p = \frac{k(k + 1)}{2}H_k - \frac{k(k-1)}{4}\]

Donde \(H_p = \sum_{i=1}^{p} \frac{1}{i} = 1 + \frac{1}{2} + \frac{1}{3}
+ ... +\frac{1}{p}\) es el p-ésimo número armónico.

\subsection{B.I}
Sea \(k = 1\):
\[\sum_{p=1}^{k} pH_p = 1*1 = 1\]
\[\frac{k(k + 1)}{2}H_k - \frac{k(k-1)}{4} = \frac{1(2)}{2} \cdot 1 - \frac{1(0)}{4} = 1\]

\subsection{H.I}
Aceptamos la propiedad:
\[\sum_{p=1}^{k} pH_p = \frac{k(k + 1)}{2}H_k - \frac{k(k-1)}{4}\]

\textbf{PDQ:}
\[\sum_{p=1}^{k+1} pH_p = \frac{(k+1)(k+2)}{2}H_{k+1} - \frac{(k+1)(k)}{4}\]

\noindent Tenemos que darnos cuenta de que:
\[\sum_{p=1}^{k+1} pH_p = \sum_{p=1}^{k} pH_p + (k+1)H_{k+1}\]

\noindent Por la propiedad sabemos que esto es igual a:
\[\frac{k(k + 1)}{2}H_k - \frac{k(k-1)}{4} + (k+1)H_{k+1}\]

\noindent También:
\[ H_{k} = \sum_{i=1}^{k+1} - \frac{1}{k+1}\]

\noindent Por lo que lo anterior quedaría:
\[\frac{k(k + 1)}{2}(\sum_{i=1}^{k+1} (\frac{1}{i}) - \frac{1}{k+1})
- \frac{k(k-1)}{4} + (k+1)\sum_{i=1}^{k+1} \frac{1}{i}\]

\noindent Ahora solo queda desarrollar:
\[\frac{k(k + 1)}{2}\sum_{i=1}^{k+1} (\frac{1}{i}) - \frac{k(k + 1)}{2(k+1)}
- \frac{k(k-1)}{4} + (k+1)\sum_{i=1}^{k+1} \frac{1}{i}\]

\[\frac{k(k + 1)}{2}\sum_{i=1}^{k+1} (\frac{1}{i})
+ (k+1)\sum_{i=1}^{k+1} (\frac{1}{i}) - \frac{k}{2} - \frac{k(k-1)}{4}\]

\[(\frac{k(k + 1)}{2} + \frac{2(k+1)}{2})\sum_{i=1}^{k+1} (\frac{1}{i})
- (\frac{2k}{4} + \frac{k(k-1)}{4})\]

\[\frac{(k + 1)(k + 2)}{2}\sum_{i=1}^{k+1} (\frac{1}{i}) - \frac{k(2 + k - 1)}{4}\]

\noindent Lo que es igual a lo que estabamos buscando:
\[\frac{(k+1)(k+2)}{2}H_{k+1} - \frac{(k+1)(k)}{4}\]
\end{document}