\documentclass[../doc.tex]{subfiles}
\begin{document}
\subsection{Primero definimos la función $Insert(L,k)$}
Usando las definiciones de listas $L \in \mathcal{L}_\mathbb{N}$ y las
funciones $Pre(L)$ y $Tail(L)$ vistas en clases, y asumiendo que $L$ está
ordenada, definiremos $Insert(L,k)$ como:

\[Insert(\emptyset,k) = \emptyset \rightarrow k\]
\[Insert(L,k) = \left\{
\begin{array}{lr}
  L \rightarrow k, & k \ge Tail(L)\\
  Insert(Pre(L),k) \rightarrow Tail(L), & k < Tail(L)
\end{array}\right .\]

Como la lista $L$ está ordenada, si $k$ es mayor o igual que $Tail(L)$,
significa que k debe ir al final de la lista. Si no, llama recursivamente a
$Insert(L,k)$ sin el último elemento, hasta que $k$ sea mayor, o se llegue al
inicio de la lista.

\subsection{Definimos $InsertSort(L \rightarrow k)$}
Usando la definición de $Insert(L,k)$ recién hecha, definiremos $InsertSort(L \rightarrow k)$ como:

\[InsertSort(\emptyset) = \emptyset \]
\[InsertSort(L \rightarrow k) = Insert(InsertSort(L),k) \]

Lo que ocurre es lo siguiente, la función $InsertSort$ continúa invocando a la
función $Insert$ hasta que la lista esté vacía. Una vez que la lista está
vacía, $InsertSort$ comienza a retornar de manera recursiva los elementos de la
lista a la función $Insert$, que los inserta en el orden correspondiente.

\subsection{Explicación paso a paso de $InsertSort(L \rightarrow k)$}
Ahora explicaremos el funcionamiento paso a paso de la función $InsertSort(L
\rightarrow k)$ recién definida, con la lista $L = \emptyset \rightarrow 5
\rightarrow 1 \rightarrow 2$. 

\textit{Primero llamamos a $ InsertSort(\emptyset \rightarrow 5 \rightarrow 1 \rightarrow 2) $ }

\[InsertSort(\emptyset \rightarrow 5 \rightarrow 1 \rightarrow 2) = Insert(InsertSort(\emptyset \rightarrow 5 \rightarrow 1 ),2) \]

\emph{Como $L$ no está vacía, se llama a $InsertSort(\emptyset \rightarrow 5 \rightarrow 1)$}

\[InsertSort(\emptyset \rightarrow 5 \rightarrow 1) = Insert(InsertSort(\emptyset \rightarrow 5),1) \]

\emph{Como $L$ no está vacía, se llama a $InsertSort(\emptyset \rightarrow 5)$}

\[InsertSort(\emptyset \rightarrow 5) = Insert(InsertSort(\emptyset),5) \]

\emph{Ahora que $L$ está vacía, las llamadas recursivas comienzan a resolverse en la stack}

\emph{Se resulve la última llamada}

\[Insert(\emptyset,5) = \emptyset \rightarrow 5\]

\emph{Se resuelve la anterior}

\[Insert(\emptyset \rightarrow 5,1) = \emptyset \rightarrow 1 \rightarrow 5\]

\emph{Se resuelve la anterior y primera llamada de la función}

\[Insert(\emptyset \rightarrow 1 \rightarrow 5,2) = \emptyset \rightarrow 1 \rightarrow 2 \rightarrow 5\]

Finalmente, $InsertSort(\emptyset \rightarrow 5 \rightarrow 1 \rightarrow 2)$
entrega la lista ordenada de manera ascendente; $L' = \emptyset \rightarrow 1
\rightarrow 2 \rightarrow 5$. 

\subsection{BONUS}
Para comprobar que InsertSort efectivamente entrega una lista ordenada,
crearemos una función $IsOrdered(L \rightarrow k)$ que lo demuestre
inductivamente. $IsOrdered(L \rightarrow k)$ recibe una lista y retorna true si
está ordenada, o false si no.

\textit{
Se considera que una lista vacía no puede ser comparada, por ello la definición
comienza con una lista de un solo elemento.}
\[IsOrdered(\rightarrow k) = true\]
\[IsOrdered(L \rightarrow k) =
\left\{
\begin{array}{lr}
  false, & k < Tail(L)\\
  IsOrdered(L), & k \ge Tail(L)
\end{array} \right .\]

Esta función pregunta recursivamente si el último elemento de la lista es mayor 
o igual que el penúltimo, si es así, sigue sigue la recursión. En caso de no 
serlo, significa que la lista no se encuentra ordenada, retornando $false$ 
inmediatamente.

Si es una lista de un solo elemento, claramente está ordenada, por lo que 
devuelve $true$.

Ahora usaremos esta función con la lista entregada por $InsertSort(L \rightarrow k)$ en la explicación anterior:
\[L = \rightarrow 1 \rightarrow 2 \rightarrow 5\]
\emph{Como k = 5 es mayor o igual que $Tail(L)$ = 2}
\[IsOrdered(\rightarrow 1 \rightarrow 2 \rightarrow 5) = IsOrdered(\rightarrow 1 \rightarrow 2)\]
\emph{Como k = 2 es mayor o igual que $Tail(L)$ = 1}
\[IsOrdered(\rightarrow 1 \rightarrow 2) = IsOrdered(\rightarrow 1)\]
\emph{Y ahora es caso base, así que: }
\[IsOrdered(\rightarrow 1) = true\]
Con esto demostramos que efectivamente la lista $L$ está ordenada de menor a mayor. 

\noindent Entonces, podemos asumir que la siguiente llamada de funciones
\[IsOrdered(InsertSort(L \rightarrow k))\]
retorna true.

\emph{Ahora demostraremos que esta llamada de funciones:}
\[IsOrdered(InsertSort(L \rightarrow k \rightarrow j))\]
\emph{también retorna true.}

\[InsertSort(L \rightarrow k \rightarrow j) = Insert(InsertSort(L \rightarrow k), j)\]
Como sabemos que $ InsertSort(L \rightarrow k) $ está ordenado, vamos a obviar esa parte.

Entonces se llama a $Insert()$, por lo que si $j$ es mayor o igual al
$Tail(L \rightarrow k)$, que es $k$, se pone al final de la lista, quedando:
\[L \rightarrow k \rightarrow j\]
Como $ InsertSort(L \rightarrow k) $ está ordenado, esta nueva lista también lo está.

Ahora si $j$ es menor que $k$ la función se llama recursivamente
hasta encontrar su posición correspondiente, dejando la lista ordenada.

Para corroborar se llamó a la función $IsSortered()$. En el primer caso donde $j$ es
mayor a $k$, se revisa que $j$ es mayor a $k$, para luego revisar si
$IsOrdered(L \rightarrow k)$ está ordenada, por la H.I sabemos que esto retorna true.

En el otro caso, donde $j$ es menor a $k$. Para simplificar digamos que el orden es el
siguiente $L \rightarrow j \rightarrow k$, la función revisaría que $j$ es menor que
$k$ por lo que ahora llama a $IsOrdered(L \rightarrow j)$, debido a que cuando se llamó a
la función $Insert(L, j)$ se tuvo que revisar que $j$ fuera mayor que el $Tail(L)$
sabemos que todavía sigue en orden. Luego se hace la llamada con $L$, pero se sabe que
si $L \rightarrow k$ está ordenada $L$ tiene que estar ordenada también, por lo que
la función terminaría retornando true.
\end{document}
