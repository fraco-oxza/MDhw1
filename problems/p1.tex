\documentclass[../doc.tex]{subfiles}
\begin{document}
\subsection{Primero definimos la función $Insert(L,k)$}
Usando las definiciones de listas $L \in \mathcal{L}_\mathbb{N}$ y las funciones $Pre(L)$ y $Tail(L)$ vistas en clases, y asumiendo que $L$ está ordenada, definiremos $Insert(L,k)$ como:

\[Insert(\emptyset,k) = \emptyset \rightarrow k\]
\[Insert(L,k)\ = \left\{\begin{array}{lr}
  L \rightarrow k, & k \ge Tail(L)\\
  Insert(Pre(L),k) \rightarrow Tail(L), & k < Tail(L)
\end{array}\right\}\]\\

Como la lista $L$ está ordenada, si $k$ es mayor o igual que $Tail(L)$, significa que k debe ir al final de la lista. Si no, llama recursivamente a $Insert(L,k)$ sin el último elemento, hasta que $k$ sea mayor, o se llegue al inicio de la lista.

\subsection{Definimos $InsertSort(L \rightarrow k)$}

\subsection{Explicación paso a paso de $InsertSort(L \rightarrow k)$}
Ahora explicaremos el funcionamiento paso a paso de la función $InsertSort(L \rightarrow k)$ recién definida, con la lista $L = \rightarrow 5 \rightarrow 1 \rightarrow 2$. 
\subsection{BONUS}
Para comprobar que InsertSort efectivamente entrega una lista ordenada, crearemos una función $IsOrdered(L \rightarrow k)$ que lo demuestre inductivamente.
$IsOrdered(L \rightarrow k)$ recibe una lista y retorna true si está ordenada, o false si no.\\

\emph{Se considera que una lista vacía no puede ser comparada, por ello la definición comienza con una lista de un solo elemento.}

\[IsOrdered(\rightarrow k) = true \]
\[IsOrdered(L \rightarrow k)\ = \left\{\begin{array}{lr}
  false, & k < Tail(L)\\
  IsOrdered(L), & k \ge Tail(L)
\end{array}\right\}\]\\

Esta función pregunta recursivamente si el último elemento de la lista es mayor o igual que el penúltimo, si es así, sigue sigue la recursión. En caso de no serlo, significa que la lista no se encuentra ordenada, retornando $false$ inmediatamente.\\
Si es una lista de un solo elemento, claramente está ordenada, así que devuelve $true$.\\

Ahora usaremos esta función con la lista entregada por $InsertSort(L \rightarrow k)$ en la explicación anterior:
\[L = \rightarrow 1 \rightarrow 2 \rightarrow 5\]
\emph{Como k = 5 es mayor o igual que Tail(L) = 2}
\[IsOrdered(\rightarrow 1 \rightarrow 2 \rightarrow 5) = IsOrdered(\rightarrow 1 \rightarrow 2)\]
\emph{Como k = 2 es mayor o igual que Tail(L) = 1}
\[IsOrdered(\rightarrow 1 \rightarrow 2) = IsOrdered(\rightarrow 1)\]
\emph{Y ahora es caso base, así que: }
\[IsOrdered(\rightarrow 1) = true\]
Con esto demostramos que efectivamente la lista $L$ está ordenada de menor a mayor. 
\end{document}
