\documentclass[../doc.tex]{subfiles}
\begin{document}
\subsection{Definición}
Considere el conjunto \(S\) de secuencias (o strings) compuestas por símbolos 0
y 1, definido por inducción de la siguiente manera:
\begin{itemize}
\item 01 es una secuencia en \(S\).
\item Si u es una secuencia en \(S\), entonces la secuencia 0\(u\)1 está en \(S\).
\item Si \(u\) y \(v\) son secuencias en \(S\), entonces la secuencia \(uv\) está en S.
\end{itemize}

\noindent Por ejemplo, las secuencias 0011 y 0101 están en S.

\subsection{Demostración de la secuencia 00011101}
Se puede demostrar que la secuencia pertenece al conjunto \(S\) usando las
reglas dadas, logrando llegar a la conclusión de que es una extensión del caso
base.

\noindent Dado que \(uv\) está en \(S\), podemos usar \(u\) como 000111 y \(v\)
como 01. \(v\) está en \(S\) porque es el caso base.

\textbf{PDQ:} \(u\) es una secuencia en \(S\).

\noindent Dado que 0\(u\)1 es una secuencia en \(S\), podemos cambiar nuestro
\(u\) a 0011. Podemos repetir esto, y ahora \(u\) sería 01, que es el caso
base.

\noindent Con esto se demuestra que la secuencia 00011101 se encuentra en el
conjunto \(S\).

\subsection{Demostración que todas las secuencias tienen igual cantidad de 0's y 1's}
\(\#0_u\) = Cantidad de 0 en la secuencia \(u\)

\subsubsection*{\emph{B.I.}}
\(u\) = 01

\noindent La secuencia \(u \in S\).

\[\#0_u = 1\ \land\ \#1_u = 1\]
\[\therefore \#0_u = \#1_u\]

\subsubsection*{\emph{H.I.}}
Asumimos que para \(u \in S\ \rightarrow\ \#0_u = \#1_u\).

\subsubsection*{\emph{T.I.}}

\textbf{PDQ:} \(\#0_{uv} = \#1_{uv}\ \land\ \#0_{0u1} = \#1_{0u1}\)
\[\#0_{0u1} = \#0_u + 1\ \land\ \#1_{0u1} = \#1_u + 1\]
\[\#0_u = \#1_u\]
\[\therefore \#0_{0u1} = \#1_{0u1}\]

\noindent Ahora falta demostrar que \(\#0_{uv} = \#1_{uv}\):
\[\#0_{uv} = \#0_u + \#0_v\ \land\ \#1_{uv} = \#1_u + \#1_v\]
\[\#0_u = \#1_u\]

\noindent Y como \(v \in S\):
\[\#0_v = \#1_v\]
\[\therefore \#0_{uv} = \#1_{uv}\]

\subsection{Demostración que 11010001 no está en \(S\)}

\subsubsection*{\emph{Propiedad}}
Toda secuencia que pertenezca a \(S\) comienza por el carácter 0.

\subsubsection*{\emph{B.I}}
La secuencia 01 comienza por 0.

\subsubsection*{\emph{H.I}}
Aceptamos la propiedad como verdadera para \(u \in S\).

\textbf{PDQ:} \(0u1\) cumple y que \(uv\) también cumple.

\noindent \(0u1\) comienza por 0 sin importar la secuencia \(u\).

\noindent En \(uv\) la secuencia \(u\) cumple la propiedad, lo que significa
que \(uv\) también cumple la propiedad.

\subsubsection*{\emph{Conclusión}}
La secuencia 11010001 no cumple esta propiedad, por tanto no pertenece al
conjunto \(S\).
\end{document}
