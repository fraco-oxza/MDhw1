\documentclass[../doc.tex]{subfiles}
\begin{document}

\subsection{Teorema y demostración}
Considere este teorema y su “demostración” por inducción:

\textbf{Teorema} Dado un conjunto de \(n\) niñas, si al menos una de ellas tiene
ojos azules, entonces las \(n\) niñas tienen ojos azules.

\textbf{Demostración.} Para n = 1 el enunciado es obviamente cierto. Supongamos
que la proposición es cierta para n niñas.
Sean \(N_1, ... , N_{n+1}\) niñas con al menos una, pongamos \(N_1\), con ojos azules.
Veamos que todas tienen los ojos azules:

El grupo de niñas \(N_1, ... , N_n\) verifica entonces la hipótesis de inducción,
con lo que todas ellas son de ojos azules. Por tanto, como \(N_2\) tiene los ojos azules,
tambien \(N_2, ... , N_{n+1}\) verifica la hipótesis de inducción, con lo que dichas niñas
y en particular \(N_{n+1}\) tiene los ojos azules. Así pues, \(N_1, ... , N_n, N_{n+1}\)
tienen los ojos azules.

\subsection{Error de la demostración}

El error de esta demostración recae en tomar cualquier conjunto con \(n\) niñas y pensar
que es igual que un conjunto diferente con la misma cantidad de niñas. La verdad es que
cada conjunto de \(n\) niñas es distinto, se pueden tener distintas niñas con
distintas características físicas.

La demostración se cae de inmediato cuando conformo un conjunto con \(n\) niñas
que tengan los ojos de color verde, y ahora saco un niña de este conjunto y hago que sea
la niña \(N_{n+1}\) del conjunto con niñas de ojos azules. Según el teorema las \(n + 1\)
niñas tienen los ojos azules, pero en realidad hay una que tiene los ojos verdes.

Por esto no se puede asumir que el conjunto conformado por las niñas \(N_1, ... ,
N_n\) es el mismo que el conjunto con las niñas \(N_2, ... , N_{n+1}\). Ya que tienen la
misma cantidad de niñas, pero estas son distintas.

Un erro mas grave 

También es posible notar el error al probar más casos bases. Cuando \(n = 1\)
es evidente que si esa niña tiene los ojos azules las \(n\) niñas tienen los ojos azules.
Sin embargo, cuando \(n = 2\) una niña puede tener los ojos azules, pero no necesariamente
ambas. Por lo cual la propiedad funciona solo cuando \(n = 1\).

\end{document}
