\documentclass[../doc.tex]{subfiles}
\begin{document}

\subsection{Teorema y demostración}
Considere este teorema y su “demostración” por inducción:

\textbf{Teorema} Dado un conjunto de \(n\) niñas, si al menos una de ellas tiene
ojos azules, entonces las \(n\) niñas tienen ojos azules.

\textbf{Demostración.} Para n = 1 el enunciado es obviamente cierto. Supongamos
que la proposición es cierta para n niñas.
Sean \(N_1, ... , N_{n+1}\) niñas con al menos una, pongamos \(N_1\), con ojos azules.
Veamos que todas tienen los ojos azules:

El grupo de niñas \(N_1, ... , N_n\) verifica entonces la hipótesis de inducción,
con lo que todas ellas son de ojos azules. Por tanto, como \(N_2\) tiene los ojos azules,
también \(N_2, ... , N_{n+1}\) verifica la hipótesis de inducción, con lo que dichas niñas
y en particular \(N_{n+1}\) tiene los ojos azules. Así pues, \(N_1, ... , N_n, N_{n+1}\)
tienen los ojos azules.

\subsection{Error de la demostración}

El error de esta demostración recae en tomar cualquier conjunto con \(n\) niñas y pensar
que es igual que un conjunto diferente con la misma cantidad de niñas. La verdad es que
cada conjunto de \(n\) niñas es distinto, se pueden tener distintas niñas con
distintas características físicas.

También es posible notar el error al probar más casos bases. Cuando \(n = 1\)
es evidente que si esa niña tiene los ojos azules las \(n\) niñas tienen los 
ojos azules. Sin embargo, cuando \(n = 2\) podemos encontrar el siguiente caso 
que desacredita al teorema. Tenemos un grupo con dos niñas, una de ojos azules y 
la otra de ojos marrón, este grupo de niñas cumple con el teorema, ya que existe
una niña de ojos azules, pero a su vez hay una niña que tiene los ojos marrones 
lo cual es una contradicción clara, lo que indica que la H.I. es incorrecta.

Sin embargo, el error más grave es que está utilizando la T.I. para demostrar la T.I.
Analicemos la H.I., que es la parte de la inducción que podemos aceptar como 
cierta sin cuestionarnos. La H.I. es que en un grupo de  $ N_1, ..., N_{n} $ todas 
las niñas tienen los ojos azules. Esta parte es correcta, el problema viene al 
decir que $ N_2, ..., N_{n+1} $ también es un grupo con todas las niñas con ojos
azules, ya que, esta es nuestra T.I., es decir, el autor no puede utilizarla dentro
de la demostración de la T.I. Lógicamente es similar a querer demostrar que 
algo es verdad diciendo que es verdad, utilizando esa verdad como argumento y nada más.

\end{document}
