\documentclass{article}
\usepackage{amssymb}
\usepackage{tikz}
\usepackage{subfiles}
\usepackage[margin=4cm]{geometry}
\setlength{\parskip}{10pt}

\title{Modelos Discretos: Respuestas tarea 1}
\author{Francisco Carvajal, Vicente Díaz, Benjamín Farías}

\begin{document}
\maketitle

\newpage
\section{Inducción Estructural}
%\subfile{problems/p1.tex} TODO: Descomentar luego de mover
\subsection{Primero definimos la función $Insert(L,k)$}
Usando las definiciones de listas $L \in \mathcal{L}_\mathbb{N}$ y las funciones $Pre(L)$ y $Tail(L)$ vistas en clases, y asumiendo que $L$ está ordenada, definiremos $Insert(L,k)$ como:

\[Insert(\emptyset,k) = \emptyset \rightarrow k\]
\[Insert(L,k)\ = \left\{\begin{array}{lr}
  L \rightarrow k, & k \ge Tail(L)\\
  Insert(Pre(L),k) \rightarrow Tail(L), & k < Tail(L)
\end{array}\right\}\]\\

Como la lista $L$ está ordenada, si $k$ es mayor o igual que $Tail(L)$, significa que k debe ir al final de la lista. Si no, llama recursivamente a $Insert(L,k)$ sin el último elemento, hasta que $k$ sea mayor, o se llegue al inicio de la lista.

\subsection{Definimos $InsertSort(L \rightarrow k)$}

\subsection{Explicación paso a paso de $InsertSort(L \rightarrow k)$}
Ahora explicaremos el funcionamiento paso a paso de la función $InsertSort(L \rightarrow k)$ recién definida, con la lista $L = \rightarrow 5 \rightarrow 1 \rightarrow 2$. 
\subsection{BONUS}
Para comprobar que InsertSort efectivamente entrega una lista ordenada, crearemos una función $IsOrdered(L \rightarrow k)$ que lo demuestre inductivamente.
$IsOrdered(L \rightarrow k)$ recibe una lista y retorna true si está ordenada, o false si no.\\

\emph{Se considera que una lista vacía no puede ser comparada, por ello la definición comienza con una lista de un solo elemento.}

\[IsOrdered(\rightarrow k) = true \]
\[IsOrdered(L \rightarrow k)\ = \left\{\begin{array}{lr}
  false, & k < Tail(L)\\
  IsOrdered(L), & k \ge Tail(L)
\end{array}\right\}\]\\

Esta función pregunta recursivamente si el último elemento de la lista es mayor o igual que el penúltimo, si es así, sigue sigue la recursión. En caso de no serlo, significa que la lista no se encuentra ordenada, retornando $false$ inmediatamente.\\
Si es una lista de un solo elemento, claramente está ordenada, así que devuelve $true$.\\

Ahora usaremos esta función con la lista entregada por $InsertSort(L \rightarrow k)$ en la explicación anterior:
\[L = \rightarrow 1 \rightarrow 2 \rightarrow 5\]
\emph{Como k = 5 es mayor o igual que Tail(L) = 2}
\[IsOrdered(\rightarrow 1 \rightarrow 2 \rightarrow 5) = IsOrdered(\rightarrow 1 \rightarrow 2)\]
\emph{Como k = 2 es mayor o igual que Tail(L) = 1}
\[IsOrdered(\rightarrow 1 \rightarrow 2) = IsOrdered(\rightarrow 1)\]
\emph{Y ahora es caso base, así que: }
\[IsOrdered(\rightarrow 1) = true\]
Con esto demostramos que efectivamente la lista $L$ está ordenada de menor a mayor. 
\newpage
\section{Inducción sobre strings}
\subfile{problems/p2.tex}


\newpage
\section{Definición inductiva de grafos}
\subfile{problems/p3.tex}

\newpage
\section{Inducción para resolver sumatorias}
\subfile{problems/p4.tex}

\section{Inducción sobre fórmulas lógicas}
% \subfile{problems/p5.tex} TODO: Descomentar luego de mover

\subsection{Tautologías usadas en la demostración}
\begin{enumerate}
  \item Transformación:  $p \rightarrow q \equiv \neg p \lor q$
  \item Distributividad: $p \lor (q \land r) \equiv (p \lor q) \land (p \lor r)$
  \item Distributividad: $p \land (q \lor r) \equiv (p \land q) \lor (p \land r)$
  \item Doble negación: $\neg\neg p \equiv p$
\end{enumerate}

\subsection{Demostración}
Usando inducción demostraremos que si $L(P)$ es un lenguaje proposicional, la fórmula $(\bigvee _{i=1}^{n} X_i) \rightarrow Y $ es lógicamente equivalente a $\bigwedge _{i=1}^{n} (X_i \rightarrow Y)$,
para todo $n \ge 1$.
\subsubsection*{\emph{B.I}}
\emph{n=1}
\[X_1 \rightarrow Y \equiv X_1 \rightarrow Y\]
\emph{n=2}
\[(X_1 \lor X_2) \rightarrow Y \qquad (X_1 \rightarrow Y) \land (X_2 \rightarrow Y) \]
\[\neg(X_1 \lor X_2) \lor Y \qquad (\neg X_1 \lor Y) \land (\neg X_2 \lor Y) \]
\[(\neg X_1 \land \neg X_2) \lor Y \quad \equiv \quad (\neg X_1 \land \neg X_2) \lor Y \]

\subsubsection*{\emph{H.I}}
\[ (\bigvee _{i=1}^{n} X_i) \rightarrow Y \equiv \bigwedge _{i=1}^{n} (X_i \rightarrow Y) \quad \forall n \ge 1\]

\subsubsection*{\emph{T.I}}
Para demostrar que lo propuesto en el enunciado se cumple, podemos usar el $PIS$ en lo siguiente:\\

\emph{Empezaremos por la fórmula lógica de la derecha}
\[ (\bigvee _{i=1}^{n+1} X_i) \rightarrow Y \equiv \bigwedge _{i=1}^{n+1} (X_i \rightarrow Y) \]

\emph{Aplicamos propiedad de la "ANDatoria"}
\[ \bigwedge _{i=1}^{n+1} (X_i \rightarrow Y) = \bigwedge _{i=1}^{n} (X_i \rightarrow Y) \land (X_{n+1} \rightarrow Y) \]

\emph{Por H.I}
\[ \bigwedge _{i=1}^{n} (X_i \rightarrow Y) \land (X_{n+1} \rightarrow Y) = ((\bigvee _{i=1}^{n} X_i) \rightarrow Y) \land (X_{n+1} \rightarrow Y) \] 

\emph{Aplicamos Transformación}
\[ ((\bigvee _{i=1}^{n} X_i) \rightarrow Y) \land (X_{n+1} \rightarrow Y) = (\neg(\bigvee _{i=1}^{n} X_i) \lor Y) \land (\neg X_{n+1} \lor Y) \]

\emph{Aplicamos Distributividad}
\[ (\neg(\bigvee _{i=1}^{n} X_i) \lor Y) \land (\neg X_{n+1} \lor Y) = (\neg(\bigvee _{i=1}^{n} X_i) \land \neg X_{n+1} ) \lor Y\]

\emph{Aplicamos Doble negación}
\[ (\neg(\bigvee _{i=1}^{n} X_i) \land \neg X_{n+1} ) \lor Y = \neg\neg(\neg(\bigvee _{i=1}^{n} X_i) \land \neg X_{n+1} ) \lor Y\] 
\[ \neg\neg(\neg(\bigvee _{i=1}^{n} X_i) \land \neg X_{n+1} ) \lor Y = \neg((\bigvee _{i=1}^{n} X_i) \lor  X_{n+1} )\lor Y \]

\emph{Aplicamos propiedad de la "ORatoria"}
\[ \neg((\bigvee _{i=1}^{n} X_i) \lor  X_{n+1}) \lor Y = \neg(\bigvee _{i=1}^{n+1} X_i) \lor Y \]

\emph{Aplicamos Transformación}
\[ \neg(\bigvee _{i=1}^{n+1} X_i) \lor Y = (\bigvee _{i=1}^{n+1} X_i) \rightarrow Y\]

\emph{Ahora es claro que:}
\[ (\bigvee _{i=1}^{n+1} X_i) \rightarrow Y \equiv \bigwedge _{i=1}^{n+1} (X_i \rightarrow Y) \]

\newpage
\section{Inducción sobre números naturales}
\subfile{problems/p6.tex}

\newpage
\section{Falacias inductivas}
\subfile{problems/p7.tex}


\end{document}
