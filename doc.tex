\documentclass{article}
\usepackage{amssymb}
\usepackage{tikz}
\usepackage{subfiles}
\usepackage[margin=4cm]{geometry}
\setlength{\parskip}{10pt}

\title{Modelos Discretos: Respuestas tarea 1}
\author{Francisco Carvajal, Vicente Díaz, Benjamín Farías}

\begin{document}
\maketitle

\newpage
\section{Inducción Estructural}
%\subfile{problems/p1.tex} TODO: Descomentar luego de mover
\subsection{Primero definimos la función $Insert(L,k)$}
Usando las definiciones de $Lista$ y $Pre(L)$ vistas en clases, y asumiendo que $L$ está ordenada, definiremos $Insert(L,k)$ como:
\[Insert(\emptyset,k) = \emptyset \rightarrow k\]
\[Insert(L,k)\ = \left\{\begin{array}{lr}
  L \rightarrow k, & k \ge Tail(L)\\
  Insert(Pre(L),k) \rightarrow Tail(L), & k < Tail(L)
\end{array}\right\}
\]

\subsection{Definimos $InsertSort(L)$}

\newpage
\section{Inducción sobre strings}
%\subfile{problems/p2.tex} TODO: Descomentar luego de mover
\subsection{Definición}
Considere el conjunto \(S\) de secuencias (o strings) compuestas por símbolos 0 y 1, definido por inducción de la siguiente manera:
\begin{itemize}
\item 01 es una secuencia en \(S\).
\item Si u es una secuencia en \(S\), entonces la secuencia 0\(u\)1 está en \(S\).
\item Si \(u\) y \(v\) son secuencias en \(S\), entonces la secuencia \(uv\) está en S.
\end{itemize}

\noindent Por ejemplo, las secuencias 0011 y 0101 están en S.

\subsection{Demostración de la secuencia 00011101}
Se puede demostrar que la secuencia pertenece al conjunto \(S\) usando las reglas dadas, logrando llegar a la conclusión de que es una extensión del caso base.

\noindent Dado que \(uv\) está en \(S\), podemos usar \(u\) como 000111 y \(v\) como 01. \(v\) está en \(S\) porque es el caso base.

\textbf{PDQ:} \(u\) es una secuencia en \(S\).

\noindent Dado que 0\(u\)1 es una secuencia en \(S\), podemos cambiar nuestro \(u\) a 0011. Podemos repetir esto, y ahora \(u\) sería 01, que es el caso base.

\noindent Con esto se demuestra que la secuencia 00011101 se encuentra en el conjunto \(S\).

\subsection{Demostración que todas las secuencias tienen igual cantidad de 0's y 1's}
\(\#0_u\) = Cantidad de 0 en la secuencia \(u\)

\subsubsection*{\emph{B.I.}}
\(u\) = 01

\noindent La secuencia \(u \in S\).

\[\#0_u = 1\ \land\ \#1_u = 1\]

\[\therefore \#0_u = \#1_u\]

\subsubsection*{\emph{H.I.}}
Asumimos que para \(u \in S\ \rightarrow\ \#0_u = \#1_u\).

\textbf{PDQ:} \(\#0_{uv} = \#1_{uv}\ \land\ \#0_{0u1} = \#1_{0u1}\)

\[\#0_{0u1} = \#0_u + 1\ \land\ \#1_{0u1} = \#1_u + 1\]

\[\#0_u = \#1_u\]

\[\therefore \#0_{0u1} = \#1_{0u1}\]

\noindent Ahora falta demostrar que \(\#0_{uv} = \#1_{uv}\):

\[\#0_{uv} = \#0_u + \#0_v\ \land\ \#1_{uv} = \#1_u + \#1_v\]

\[\#0_u = \#1_u\]

\noindent Y como \(v \in S\):

\[\#0_v = \#1_v\]

\[\therefore \#0_{uv} = \#1_{uv}\]

\subsection{Demostración que 11010001 no está en \(S\)}

\subsubsection*{\emph{Propiedad}}
Toda secuencia que pertenezca a \(S\) comienza por el carácter 0.

\subsubsection*{\emph{B.I}}
La secuencia 01 comienza por 0.

\subsubsection*{\emph{H.I}}
Aceptamos la propiedad como verdadera para \(u \in S\).

\textbf{PDQ:} \(0u1\) cumple y que \(uv\) también cumple.

\noindent \(0u1\) comienza por 0 sin importar la secuencia \(u\).

\noindent En \(uv\) la secuencia \(u\) cumple la propiedad, lo que significa que \(uv\) también cumple la propiedad.

\subsubsection*{\emph{Conclusión}}
La secuencia 11010001 no cumple esta propiedad, por tanto no pertenece al conjunto \(S\).

\newpage
\section{Definición inductiva de grafos}
\subfile{problems/p3.tex}

\newpage
\section{Inducción para resolver sumatorias}
\subfile{problems/p4.tex}

\section{Inducción sobre fórmulas lógicas}
% \subfile{problems/p5.tex} TODO: Descomentar luego de mover

\subsection{Tautologías usadas en la demostración}
\begin{enumerate}
  \item Transformación:  $p \rightarrow q \equiv \neg p \lor q$
  \item Distributividad: $p \lor (q \land r) \equiv (p \lor q) \land (p \lor r)$
  \item Distributividad: $p \land (q \lor r) \equiv (p \land q) \lor (p \land r)$
  \item Doble negación: $\neg\neg p \equiv p$
\end{enumerate}

\subsection{Demostración}
Usando inducción demostraremos que si $L(P)$ es un lenguaje proposicional, la fórmula $(\bigvee _{i=1}^{n} X_i) \rightarrow Y $ es lógicamente equivalente a $\bigwedge _{i=1}^{n} (X_i \rightarrow Y)$,
para todo $n \ge 1$.
\subsubsection*{\emph{B.I}}
\emph{n=1}
\[X_1 \rightarrow Y \equiv X_1 \rightarrow Y\]
\emph{n=2}
\[(X_1 \lor X_2) \rightarrow Y \qquad (X_1 \rightarrow Y) \land (X_2 \rightarrow Y) \]
\[\neg(X_1 \lor X_2) \lor Y \qquad (\neg X_1 \lor Y) \land (\neg X_2 \lor Y) \]
\[(\neg X_1 \land \neg X_2) \lor Y \quad \equiv \quad (\neg X_1 \land \neg X_2) \lor Y \]

\subsubsection*{\emph{H.I}}
\[ (\bigvee _{i=1}^{n} X_i) \rightarrow Y \equiv \bigwedge _{i=1}^{n} (X_i \rightarrow Y) \]

\subsubsection*{\emph{T.I}}
Para demostrar que lo propuesto en el enunciado se cumple, podemos usar el $PIS$ en lo siguiente:\\
\emph{Empezaremos por la fórmula lógica de la derecha}
\[ (\bigvee _{i=1}^{n+1} X_i) \rightarrow Y \equiv \bigwedge _{i=1}^{n+1} (X_i \rightarrow Y) \]

\emph{Aplicamos propiedad de la "ANDatoria"}
\[ \bigwedge _{i=1}^{n+1} (X_i \rightarrow Y) = \bigwedge _{i=1}^{n} (X_i \rightarrow Y) \land (X_{n+1} \rightarrow Y) \]

\emph{Por H.I}
\[ \bigwedge _{i=1}^{n} (X_i \rightarrow Y) \land (X_{n+1} \rightarrow Y) = ((\bigvee _{i=1}^{n} X_i) \rightarrow Y) \land (X_{n+1} \rightarrow Y) \] 

\emph{Aplicamos Transformación}
\[ ((\bigvee _{i=1}^{n} X_i) \rightarrow Y) \land (X_{n+1} \rightarrow Y) = (\neg(\bigvee _{i=1}^{n} X_i) \lor Y) \land (\neg X_{n+1} \lor Y) \]

\emph{Aplicamos Distributividad}
\[ (\neg(\bigvee _{i=1}^{n} X_i) \lor Y) \land (\neg X_{n+1} \lor Y) = (\neg(\bigvee _{i=1}^{n} X_i) \land \neg X_{n+1} ) \lor Y\]

\emph{Aplicamos Doble negación}
\[ (\neg(\bigvee _{i=1}^{n} X_i) \land \neg X_{n+1} ) \lor Y = \neg\neg(\neg(\bigvee _{i=1}^{n} X_i) \land \neg X_{n+1} ) \lor Y\] 
\[ \neg\neg(\neg(\bigvee _{i=1}^{n} X_i) \land \neg X_{n+1} ) \lor Y = \neg((\bigvee _{i=1}^{n} X_i) \lor  X_{n+1} )\lor Y \]

\emph{Aplicamos propiedad de la "ORatoria"}
\[ \neg((\bigvee _{i=1}^{n} X_i) \lor  X_{n+1}) \lor Y = \neg(\bigvee _{i=1}^{n+1} X_i) \lor Y \]

\emph{Aplicamos Transformación}
\[ \neg(\bigvee _{i=1}^{n+1} X_i) \lor Y = (\bigvee _{i=1}^{n+1} X_i) \rightarrow Y\]

\emph{Ahora es claro que:}
\[ (\bigvee _{i=1}^{n+1} X_i) \rightarrow Y \equiv \bigwedge _{i=1}^{n+1} (X_i \rightarrow Y) \]

\newpage
\section{Inducción sobre números naturales}
\subfile{problems/p6.tex}

\newpage
\section{Falacias inductivas}
% \subfile{problems/p7.tex} TODO: Descomentar esto luego de mover

\subsection{Teorema y demostración}
Considere este teorema y su “demostración” por inducción:

\noindent \textbf{Teorema} Dado un conjunto de \(n\) niñas, si al menos una de ellas tiene ojos azules, entonces las \(n\) niñas tienen ojos azules.

\noindent \textbf{Demostración.} Para n = 1 el enunciado es obviamente cierto. Supongamos que la proposición es cierta para n niñas.
Sean \(N_1, ... , N_{n+1}\) niñas con al menos una, pongamos \(N_1\), con ojos azules. Veamos que todas tienen los ojos azules:

\noindent El grupo de niñas \(N_1, ... , N_n\) verifica entonces la hipótesis de inducción, con lo que todas ellas son de ojos azules. Por tanto, como \(N_2\) tiene los ojos azules, tambien \(N_2, ... , N_{n+1}\) verifica la hipótesis de inducción, con lo que dichas niñas y en particular \(N_{n+1}\) tiene los ojos azules. Así pues, \(N_1, ... , N_n, N_{n+1}\) tienen los ojos azules.

\subsection{Error de la demostración}
El error de esta demostración recae en tomar cualquier conjunto con \(n\) niñas y pensar que es igual que otro conjunto con \(n\) niñas. La verdad es que cada conjunto de \(n\) niñas es distinto, ya que pueden tener distintas niñas con distintas características físicas.

\noindent La demostración se cae de inmediato cuando conformo un conjunto con \(n\) niñas que tengan los ojos de color verde, y ahora saco un niña de este conjunto y hago que sea la niña \(N_{n+1}\) del conjunto con niñas de ojos azules. Según el teorema las \(n + 1\) niñas tienen los ojos azules, pero en realidad hay una que tiene los ojos verdes.

\noindent Por esto no se puede asumir que el conjunto conformado por las niñas \(N_1, ... , N_n\) es el mismo que el conjunto con las niñas \(N_2, ... , N_{n+1}\).

\end{document}
